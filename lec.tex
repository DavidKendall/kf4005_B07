\documentclass[hyperref={pdfpagelabels=false},svgnames,xcolor=table]{beamer}
\mode<presentation>{%
  \usetheme{Madrid}
  \setbeamertemplate{navigation symbols}{}
}
\usepackage[english]{babel}
\usepackage[latin1]{inputenc}
\usepackage{times}
\usepackage[T1]{fontenc}
\usepackage{graphicx}
\usepackage{tikz}

\title{Operating systems fundamentals - B07}
\subtitle{}
\author{David Kendall}
\institute{Northumbria University}
\date{}
\begin{document}

\begin{frame}
\titlepage%
\end{frame}

\begin{frame}
\frametitle{What is SQL?}
\begin{itemize}
  \item Structured Query Language
  \item Used to `talk' to a database server
  \item A standard language for querying and manipulating data
  \item Used as a front-end to many database management systems, e.g.
    MySQL, Oracle, PostgreSQL, Sybase, \ldots
  \item SQL is a very high-level programming language (declarative)
  \item Three main parts: data definition language, data manipulation 
    language, user privileges
\end{itemize}
\end{frame}

\begin{frame}
  \frametitle{When do you need a database?}
  \begin{itemize}
    \item Multiple simultaneous changes to data (concurrency)
    \item Data changes frequently
    \item Large data sets where you need to be able to select items
      of current interest
    \item Share large data set among many users
    \item Rapid queries
    \item Web interface to data, especially dynamic data
  \end{itemize}
\end{frame}

\begin{frame}
  \frametitle{SQL features}
    \begin{description}
      \item[data definition language] \mbox{\strut}
        \begin{itemize}
          \item define relational schemata 
          \item create/alter/delete tables and their attributes
        \end{itemize}
      \item[data manipulation language] \mbox{\strut}
        \begin{itemize}
          \item insert/delete/modify tuples in tables
          \item query one or more tables
        \end{itemize}
      \item[user privileges] \mbox{\strut}
        \begin{itemize}
          \item who can do what to which?
        \end{itemize}
    \end{description}
\end{frame}

\begin{frame}
  \frametitle{Ways to use SQL}
  \begin{itemize}
    \item Console command (\textbf{mysql -u} \emph{user} \textbf{-p})
    \item GUI interfaces are often available
    \item Interfaces to many programming languages, e.g. Python, Perl, PHP,
      R, \ldots
    \item SQLite --- use SQL without a database server
    \item We'll use MySQL with commands from the console in this module
    \item MySQL is the most widely used open-source DataBase Management 
      System (DBMS) available today
  \end{itemize}
\end{frame}

\begin{frame}
  \frametitle{Some basic database concepts}
  \begin{itemize}
    \item A database server can contain many databases
    \item Databases are collections of tables
    \item Tables are two-dimensional with rows (tuples, observations)
      and columns (attributes, variables)
    \item Limited mathematical and summary operations available
    \item Very good at combining information from several tables
  \end{itemize}
\end{frame}

\begin{frame}[fragile]
  \frametitle{Basic interactions with the database server}
  \begin{itemize}
    \item Login \\
      \verb'mysql -u root -p' \\
      \verb'Enter password: '
    \item What databases is the server managing? \\
      \verb'SHOW DATABASES;'
    \item What tables are there in a database? \\
      \verb'SHOW TABLES in <database name>;'
    \item What columns are there in a table? \\
      \verb'SHOW COLUMNS in <table name>;'
    \item What are the types of data in a table? \\
      \verb'DESCRIBE <table name>;'
    \item How do I create a database? \\
      \verb'CREATE DATABASE <database name>';
    \item How do I select a particular database to work on? \\
      \verb'USE <database name>';
  \end{itemize}
\end{frame}

\begin{frame}[fragile]
  \frametitle{Tables in SQL}
  \begin{itemize}
    \item Assume we have a database called \verb'menagerie' that contains
      a table called \verb'pet'
  \end{itemize}
  \textbf{pet} \\
  \begin{tabular}{|c|c|c|c|c|c|}
    \hline
    \textbf{name} & \textbf{owner} & \textbf{species} & \textbf{sex} & \textbf{birth} & \textbf{death} \\
    \hline
    Fluffy & Harold & cat & f & 1993-02-04 & \\
    \hline
    Claws & Gwen & cat & m & 1994-03-17 & \\
    \hline
    Slim & Benny & snake & m & 1996-04-29 & 1997-03-11 \\
    \hline
    Buffy & Harold & dog & f & 1989-05-13 & \\
    \hline
  \end{tabular}
  \bigskip
  \begin{itemize}
    \item A \textbf{table} (or \textbf{relation}) is a \emph{multiset} of
      \emph{tuples} having the \emph{attributes} specified by the
      \emph{schema}
    \item We need to break this definition down \ldots
  \end{itemize}
\end{frame}

\begin{frame}
  \frametitle{Tables in SQL}
  \textbf{pet} \\
  \begin{tabular}{|c|c|c|c|c|c|}
    \hline
    \textbf{name} & \textbf{owner} & \textbf{species} & \textbf{sex} & \textbf{birth} & \textbf{death} \\
    \hline
    Fluffy & Harold & cat & f & 1993-02-04 & \\
    \hline
    Claws & Gwen & cat & m & 1994-03-17 & \\
    \hline
    Slim & Benny & snake & m & 1996-04-29 & 1997-03-11 \\
    \hline
    Buffy & Harold & dog & f & 1989-05-13 & \\
    \hline
  \end{tabular}
  \bigskip
  \begin{itemize}
  \item A \textbf{multiset} is an unordered list (a set with duplicate elements
    allowed)
      \begin{itemize}
        \item List: $[1, 1, 2, 3]$
        \item Set: $\{1, 2, 3\}$
        \item Multiset: $\{1, 1, 2, 3\}$
      \end{itemize}
  \end{itemize}
\end{frame}

\begin{frame}
  \frametitle{Tables in SQL}
  \textbf{pet} \\
  \begin{tabular}{|c|>{\columncolor{gray}}c|c|c|c|c|}
    \hline
    \textbf{name} & \textbf{owner} & \textbf{species} & \textbf{sex} & \textbf{birth} & \textbf{death} \\
    \hline
    Fluffy & Harold & cat & f & 1993-02-04 & \\
    \hline
    Claws & Gwen & cat & m & 1994-03-17 & \\
    \hline
    Slim & Benny & snake & m & 1996-04-29 & 1997-03-11 \\
    \hline
    Buffy & Harold & dog & f & 1989-05-13 & \\
    \hline
  \end{tabular}
  \bigskip
  \begin{itemize}
    \item An \textbf{attribute} (or \textbf{column}) is a typed data entry,
      present in each tuple in the relation
    \item Attributes must have an \emph{atomic} data type in standard SQL, 
      e.g.\ a number, a string, a date, \ldots but \emph{not} a list,
      set, table, \ldots
    \item Each value in the table must have the correct type for the
      attribute, or it may be \textbf{NULL}, indicating that the value is
      not known or not available, e.g. the value of the \textbf{death} 
      attribute for \textbf{Fluffy} is \textbf{NULL} (presumably because
      Fluffy is still alive)
    \item The number of attributes in a table gives the table's 
      \textbf{arity}
  \end{itemize}
\end{frame}

\begin{frame}
  \frametitle{Tables in SQL}
  \textbf{pet} \\
  \begin{tabular}{|c|c|c|c|c|c|}
    \hline
    \textbf{name} & \textbf{owner} & \textbf{species} & \textbf{sex} & \textbf{birth} & \textbf{death} \\
    \hline
    Fluffy & Harold & cat & f & 1993-02-04 & \\
    \hline
    Claws & Gwen & cat & m & 1994-03-17 & \\
    \hline
    \rowcolor{gray}Slim & Benny & snake & m & 1996-04-29 & 1997-03-11 \\
    \hline
    Buffy & Harold & dog & f & 1989-05-13 & \\
    \hline
  \end{tabular}
  \bigskip
  \begin{itemize}
    \item A \textbf{tuple} (or \textbf{row}) is a single entry in the 
      table, having the attributes specified by the schema
    \item Also sometimes referred to as a \textbf{record}
    \item The number of rows in a table gives the table's \textbf{cardinality}
  \end{itemize}
\end{frame}

\begin{frame}
  \frametitle{Data types in SQL}
  \begin{itemize}
    \item Atomic types:
      \begin{itemize}
        \item Characters: \textbf{CHAR(20)}, \textbf{VARCHAR(50)}
        \item Numbers: \textbf{INT}, \textbf{BIGINT}, \textbf{SMALLINT},
          \textbf{FLOAT}
        \item Others: \textbf{MONEY}, \textbf{DATE}, \textbf{DATETIME}
      \end{itemize}
    \item Every attribute must have an atomic type, i.e. tables are \emph{flat}
  \end{itemize}
\end{frame}

\begin{frame}[fragile]
  \frametitle{Table schemas and table creation}
  \begin{itemize}
    \item The \textbf{schema} of a table is its name, its attributes, and
      their types
    \item The table schema is specified when the table is created, e.g.\ 
\begin{verbatim}
mysql> CREATE TABLE pet (name VARCHAR(20), 
    -> owner VARCHAR(20), species VARCHAR(20), 
    -> sex CHAR(1), birth DATE, death DATE);
\end{verbatim}
  \end{itemize}
\end{frame}

\begin{frame}
  \frametitle{Keys and Key constraints}
  \begin{itemize}
    \item A \emph{key} is a \emph{minimal subset of attributes} that
      acts as a unique identifier for a tuple in a relation
    \item A key is an implicit constraint on which tuples can be in the 
      relation 
    \item If two tuples agree on the value of the key, they must be the
      same tuple!
    \item A possible key for \textbf{pet} is underlined below, assuming that
      there is no pet with the same name, owner, and birth date 
  \end{itemize}
  \begin{block}{pet key}
    pet(\underline{name} VARCHAR(20), \underline{owner} VARCHAR(20), species VARCHAR(20), sex CHAR(1),
        \underline{birth} DATE, death DATE)
  \end{block}
\end{frame}

\begin{frame}[shrink=10, fragile]
  \frametitle{SQL queries --- SELECT}
  \begin{itemize}
    \item Basic form (there are many more variations)
      \begin{tabular}{>{\color{blue}}ll}
        SELECT & <attributes> \\
        FROM & <one or more tables> \\
        WHERE & <conditions>
      \end{tabular}
    \item This is known as an \textbf{SFW} query, e.g.\ 
  \begin{verbatim}
  mysql> SELECT name, species, birth
      -> FROM pet
      -> WHERE owner = 'harold';
+--------+---------+------------+
| name   | species | birth      |
+--------+---------+------------+
| Fluffy | cat     | 1993-02-04 |
| Buffy  | dog     | 1989-05-13 |
+--------+---------+------------+
  \end{verbatim}
\item Note it is the final semi-colon (\textbf{;}) that causes the
  command to be executed --- this is generally the case when entering
  SQL commands
  \end{itemize}
\end{frame}

\begin{frame}[fragile]
  \frametitle{SQL queries --- SELECT}
  \begin{itemize}
    \item You can select \emph{ALL} attributes from a table
      using \textbf{*}, e.g.
\begin{scriptsize}
\begin{verbatim}
mysql> SELECT *
    -> FROM pet
    -> WHERE birth > '1994-01-01';
+-------+-------+---------+------+------------+------------+
| name  | owner | species | sex  | birth      | death      |
+-------+-------+---------+------+------------+------------+
| Claws | Gwen  | cat     | m    | 1994-03-17 | NULL       |
| Slim  | Benny | snake   | m    | 1996-04-29 | 1997-03-11 |
+-------+-------+---------+------+------------+------------+
\end{verbatim}
\end{scriptsize}
  \end{itemize}
\end{frame}

\begin{frame}
  \frametitle{Some syntax details (UNIX environment)}
  \begin{itemize}
    \item SQL commands and attributes are \emph{not} case-sensitive, e.g.\ 
      \begin{itemize}
        \item Same: SELECT, Select, select
        \item Same: CREATE, Create, crEAtE
        \item Same: BIRTH, Birth, birth
      \end{itemize}
    \item Names of databases, tables, and values \emph{are} case-sensitive, 
      e.g.\ 
      \begin{itemize}
        \item Different: PET, Pet, pet
        \item Different: 'HAROLD', 'Harold', 'harold'
      \end{itemize}
    \item A common convention is to use UPPERCASE letters for SQL commands,
      e.g. SELECT, FROM, WHERE, CREATE and lowercase letter for the names
      of databases, tables, attributes \ldots e.g.\ pet, owner
    \item Use single quotes for constants, e.g.\ \\
      \begin{tabular}{ll}
        'abc' & YES \\
        "abc" & NO
      \end{tabular}
  \end{itemize}
\end{frame}

\begin{frame}[fragile]
  \frametitle{Exercise --- SELECT}
  \begin{itemize}
    \item Write a query to show the name and species of all female pets
    \item<2-> 
\begin{verbatim}
mysql> SELECT name, species
    -> FROM pet
    -> WHERE sex = 'f';
+--------+---------+
| name   | species |
+--------+---------+
| Fluffy | cat     |
| Buffy  | dog     |
+--------+---------+
\end{verbatim}
  \end{itemize}
\end{frame}

\begin{frame}[fragile]
  \frametitle{INSERT command}
  \begin{itemize}
    \item The \textbf{INSERT} command is used to add a row to a table, e.g.\ 
\begin{scriptsize}
\begin{verbatim}
mysql> INSERT INTO pet
    -> VALUES ('Puffball', 'Diane', 'hamster', 'f', 
    -> '1999-03-30', NULL);
Query OK, 1 row affected (0.04 sec)

mysql> SELECT * FROM pet;
+----------+--------+---------+------+------------+------------+
| name     | owner  | species | sex  | birth      | death      |
+----------+--------+---------+------+------------+------------+
| Fluffy   | Harold | cat     | f    | 1993-02-04 | NULL       |
| Claws    | Gwen   | cat     | m    | 1994-03-17 | NULL       |
| Slim     | Benny  | snake   | m    | 1996-04-29 | 1997-03-11 |
| Buffy    | Harold | dog     | f    | 1989-05-13 | NULL       |
| Puffball | Diane  | hamster | f    | 1999-03-30 | NULL       |
+----------+--------+---------+------+------------+------------+
\end{verbatim}
\end{scriptsize}
  \end{itemize}
\end{frame}

\begin{frame}[shrink=10,fragile]
  \frametitle{ALTER TABLE and UPDATE commands}
  \begin{itemize}
    \item The \textbf{ALTER TABLE} command can be used to add a
      column to a table, e.g.\
\begin{scriptsize}
\begin{verbatim}

mysql> ALTER TABLE pet ADD COLUMN weight INT;
Query OK, 0 rows affected (0.63 sec)
Records: 0  Duplicates: 0  Warnings: 0

mysql> SELECT * FROM pet;
+----------+--------+---------+------+------------+------------+--------+
| name     | owner  | species | sex  | birth      | death      | weight |
+----------+--------+---------+------+------------+------------+--------+
| Fluffy   | Harold | cat     | f    | 1993-02-04 | NULL       |   NULL |
| Claws    | Gwen   | cat     | m    | 1994-03-17 | NULL       |   NULL |
| Slim     | Benny  | snake   | m    | 1996-04-29 | 1997-03-11 |   NULL |
| Buffy    | Harold | dog     | f    | 1989-05-13 | NULL       |   NULL |
| Puffball | Diane  | hamster | f    | 1999-03-30 | NULL       |   NULL |
+----------+--------+---------+------+------------+------------+--------+

\end{verbatim}
\end{scriptsize}
    \item The \textbf{UPDATE} command can be used to update the value of
      an entry, e.g.\
\begin{scriptsize}
\begin{verbatim}

mysql> UPDATE pet SET weight = 2345 WHERE name = 'Fluffy';
Query OK, 1 row affected (0.04 sec)
Rows matched: 1  Changed: 1  Warnings: 0
\end{verbatim}
\end{scriptsize}
  \end{itemize}
\end{frame}

\begin{frame}[shrink=10,fragile]
  \frametitle{New data set}
  \begin{itemize}
    \item We can update our pet database with the weights of our pets,
      ending up with, e.g.\ 
      \begin{scriptsize}
\begin{verbatim}

mysql> SELECT * FROM pet;
+----------+--------+---------+------+------------+------------+--------+
| name     | owner  | species | sex  | birth      | death      | weight |
+----------+--------+---------+------+------------+------------+--------+
| Fluffy   | Harold | cat     | f    | 1993-02-04 | NULL       |   2345 |
| Claws    | Gwen   | cat     | m    | 1994-03-17 | NULL       |   4321 |
| Slim     | Benny  | snake   | m    | 1996-04-29 | 1997-03-11 |  10123 |
| Buffy    | Harold | dog     | f    | 1989-05-13 | NULL       |   6543 |
| Puffball | Diane  | hamster | f    | 1999-03-30 | NULL       |     45 |
+----------+--------+---------+------+------------+------------+--------+

\end{verbatim}
      \end{scriptsize}
    \item We're assuming here that all weights are expressed as an integer 
      number of grams
    \item We'll use this table in our examples from now on
  \end{itemize}
\end{frame}

\begin{frame}
  \frametitle{Aggregating data}
  \begin{itemize}
    \item Sometimes we want a summary of the data rather than the details 
    \item SQL has several aggregate functions to help with this, e.g. \\
      \begin{tabular}{>{\color{blue}}lp{0.8\textwidth}}
        COUNT & counts the number of rows in a table or view \\
        AVG & calculates the average of a set of values \\
        MIN & gets the minimum value in a set of values \\
        MAX & gets the maximum value in a set of values \\
        SUM & calculates the sum of values
      \end{tabular}
    \item Examples to follow \ldots
  \end{itemize}
\end{frame}

\begin{frame}[shrink=10,fragile]
  \frametitle{Aggregate function examples}
  \begin{itemize}
    \item {\color{blue}{COUNT}}
\begin{verbatim}
mysql> SELECT COUNT(owner) FROM pet;
+--------------+
| COUNT(owner) |
+--------------+
|            5 |
+--------------+

\end{verbatim}
    \item Why is the answer 5? 
    \item What if we want the number of \emph{different}
      owners?
\begin{verbatim}
mysql> SELECT COUNT(DISTINCT owner) FROM pet;
+-----------------------+
| COUNT(DISTINCT owner) |
+-----------------------+
|                     4 |
+-----------------------+
\end{verbatim}
  \end{itemize}
\end{frame}

\begin{frame}[shrink=10,fragile]
  \frametitle{Aggregate function examples}
  \begin{itemize}
    \item {\color{blue}{AVG}}
\begin{verbatim}
mysql> SELECT AVG(weight) FROM pet;
+-------------+
| AVG(weight) |
+-------------+
|   4675.4000 |
+-------------+

\end{verbatim}
    \item Use {\color{blue}{GROUP BY}} to average over an attribute
\begin{verbatim}

mysql> SELECT species, AVG(weight) 
    -> FROM pet GROUP BY species;
+---------+-------------+
| species | AVG(weight) |
+---------+-------------+
| cat     |   3333.0000 |
| dog     |   6543.0000 |
| hamster |     45.0000 |
| snake   |  10123.0000 |
+---------+-------------+
\end{verbatim}
  \end{itemize}
\end{frame}

\begin{frame}[shrink=10,fragile]
  \frametitle{Aggregate function examples}
  \begin{itemize}
    \item {\color{blue}{MIN}}
\begin{verbatim}
mysql> SELECT MIN(weight) FROM pet;
+-------------+
| MIN(weight) |
+-------------+
|          45 |
+-------------+

\end{verbatim}
    \item Use {\color{blue}{GROUP BY}} to get the minimum over an attribute
\begin{verbatim}
mysql> SELECT owner, MIN(weight) FROM pet GROUP BY owner;
+--------+-------------+
| owner  | MIN(weight) |
+--------+-------------+
| Benny  |       10123 |
| Diane  |          45 |
| Gwen   |        4321 |
| Harold |        2345 |
+--------+-------------+
\end{verbatim}
  \end{itemize}
\end{frame}

\begin{frame}[shrink=10,fragile]
  \frametitle{Aggregate function examples}
  \begin{itemize}
    \item {\color{blue}{SUM}}
\begin{verbatim}

mysql> SELECT SUM(weight) FROM pet;
+-------------+
| SUM(weight) |
+-------------+
|       23377 |
+-------------+

\end{verbatim}
    \item Use {\color{blue}{GROUP BY}} to get the sum over an attribute
\begin{verbatim}

mysql> SELECT owner, SUM(weight) FROM pet GROUP BY owner;
+--------+-------------+
| owner  | SUM(weight) |
+--------+-------------+
| Benny  |       10123 |
| Diane  |          45 |
| Gwen   |        4321 |
| Harold |        8888 |
+--------+-------------+
\end{verbatim}
  \end{itemize}
\end{frame}

\begin{frame}[fragile]
  \frametitle{Exercise}
  \begin{itemize}
    \item Write a query to list every species and the total weight of pets of
      that species.
    \item<2-> Answer
\begin{verbatim}
mysql> SELECT species, SUM(weight) 
    -> FROM pet GROUP BY species;
+---------+-------------+
| species | SUM(weight) |
+---------+-------------+
| cat     |        6666 |
| dog     |        6543 |
| hamster |          45 |
| snake   |       10123 |
+---------+-------------+
\end{verbatim}
  \end{itemize}
\end{frame}

\begin{frame}[fragile]
  \frametitle{Ordering data}
  \begin{itemize}
    \item If we want to order the results of our last
      exercise by total weight, we can do it like this \ldots
    \item {\color{blue} ORDER BY}
\begin{verbatim}
mysql> SELECT species, SUM(weight)
    -> FROM pet GROUP BY species 
    -> ORDER BY SUM(weight);
+---------+-------------+
| species | SUM(weight) |
+---------+-------------+
| hamster |          45 |
| dog     |        6543 |
| cat     |        6666 |
| snake   |       10123 |
+---------+-------------+
\end{verbatim}
  \item Add \verb'DESC' after \verb'ORDER BY SUM(weight)' to get the
    results in descending order of weight
  \end{itemize}
\end{frame}

\begin{frame}[fragile]
  \frametitle{Getting data from multiple tables}
  \begin{itemize}
    \item Suppose we decide that we want to record information about
      special events in the lives of our pets.
    \item We create a new table called \verb'event' to store this
      information, e.g.\ 
      \begin{scriptsize}
\begin{verbatim}
mysql> SELECT * FROM event;
+--------+------------+----------+-----------------------------+
| name   | date       | type     | remark                      |
+--------+------------+----------+-----------------------------+
| Fluffy | 1995-05-15 | litter   | 4 kittens, 3 female, 1 male |
| Claws  | 1995-09-27 | litter   | 6 kittens, 2 female, 4 male |
| Claws  | 1998-03-17 | birthday | new flea collar             |
| Buffy  | 1993-06-23 | litter   | 5 puppies, 2 female, 3 male |
| Buffy  | 1994-06-19 | litter   | 3 puppies, 3 female         |
| Slim   | 1993-08-03 | vet      | swallowed tennis ball       |
+--------+------------+----------+-----------------------------+

\end{verbatim}
      \end{scriptsize}
    \item What can we do if we need a list of all those owners whose
      pets have had a litter?
  \end{itemize}
\end{frame}

\begin{frame}[fragile]
  \frametitle{Getting data from multiple tables}
  \begin{itemize}
    \item The litter information is in the event table
    \item The owner information is in the pet table
    \item We need both tables to get the information about owners
      whose pets have had a litter
      \begin{scriptsize}
\begin{verbatim}
mysql> SELECT owner 
    -> FROM pet JOIN event 
    -> ON (pet.name = event.name) 
    -> WHERE event.type = 'litter';
+--------+
| owner  |
+--------+
| Harold |
| Gwen   |
| Harold |
| Harold |
+--------+
\end{verbatim}
      \end{scriptsize}
    \item Notice the use of the keyword {\color{blue}{JOIN}} to combine the
      data from the pet and the event tables (giving the 
      {\color{blue}{}CROSS PRODUCT})
    \item The {\color{blue}{ON}} clause ensures that we only have rows in
      the result where the pet name is the same as the event name
  \end{itemize}
\end{frame}

\begin{frame}[fragile]
  \frametitle{Getting data from multiple tables}
  \begin{itemize}
    \item What if we just want a list of all the \emph{different} 
      owners whose pets have had a litter?
      \begin{scriptsize}
\begin{verbatim}
mysql> SELECT DISTINCT owner 
    -> FROM pet JOIN event 
    -> ON (pet.name = event.name) 
    -> WHERE event.type = 'litter';
+--------+
| owner  |
+--------+
| Harold |
| Gwen   |
+--------+

\end{verbatim}
      \end{scriptsize}
    \item Notice the use of the keyword {\color{blue}{DISTINCT}}
  \end{itemize}
\end{frame}

\begin{frame}[shrink=10,fragile]
  \frametitle{Using sub-queries}
  \begin{itemize}
    \item A query returns a view (table) as its result
    \item The result can be used like a table in a larger query
    \item E.g. to find the owners and the names of their lightest pet
      to have had a litter, we can try \ldots
      \begin{scriptsize}
\begin{verbatim}
mysql> SELECT pet.owner, pet.name, weight, type 
    -> FROM pet JOIN event ON (pet.name = event.name) 
    ->          JOIN (SELECT owner, MIN(weight) AS minw 
    ->                FROM pet GROUP BY owner) weights 
    ->          ON (weights.owner = pet.owner AND weights.minw = pet.weight) 
    -> WHERE event.type = 'litter';
+--------+--------+--------+--------+
| owner  | name   | weight | type   |
+--------+--------+--------+--------+
| Harold | Fluffy |   2345 | litter |
+--------+--------+--------+--------+
\end{verbatim}
      \end{scriptsize}
    \item Notice how we are able to give a name, \verb'weights', to the
      result of the sub-query and then to use it in the specification of a
      JOIN
    \item This is really beyond the scope of this short introduction to
      SQL but points the way to what is possible
  \end{itemize}
\end{frame}

\begin{frame}
  \frametitle{Acknowledgments}
  \begin{itemize}
    \item The examples here are based on the examples in 
      the \href{https://dev.mysql.com/doc/refman/5.7/en/tutorial.html}{\color{blue}{MySQL tutorial}}
    \item Some of the other material is adapted from
      \begin{itemize}
        \item Spector, P., \href{https://www.stat.berkeley.edu/~spector/sql.pdf}{\color{blue}Introduction to SQL}, Slides, University of
          California, Berkeley
        \item Bailis, P., \href{http://web.stanford.edu/class/cs145/}{\color{blue}CS145: Introduction to Databases}, Stanford University
      \end{itemize}
  \end{itemize}
\end{frame}

\end{document}
\begin{frame}
  \frametitle{}
  \begin{itemize}
    \item
  \end{itemize}
\end{frame}
